\chapter{自抗扰控制技术分析}
	控制是以适当的控制力来驾驭被控制对,使其运动在各种扰动作用下也能按期望的方式变化。施加控制力的根本途径和目的是“感受控制目标与对象实际行为之间的误差,适当处理这个误差来消除误差”。近一个世纪的控制理论发展的历史就是围绕“消除这个误差”的两种不同方法相互交错而发展的历史。

\section{自抗扰控制技术}
\subsection{跟踪微分器}

\subsection{扩张状态观测器}

\subsection{非线性状态误差反馈}

\section{自抗扰控制技术的问题}

\subsection{}

\subsection{}